% vim: tw=99

\documentclass{article}

\usepackage{fancyhdr} \usepackage{hyperref} \usepackage{listings} \usepackage{minted}

\pagestyle{fancy}

\begin{document}

\title{ScPy: A SuperCollider Extension for Performing Numerical Computation via Embedded Python}

\author{Noah Weninger \and Abram Hindle}

\maketitle

\begin{abstract}

    SuperCollider, a language for sound synthesis and algorithmic composition of audio, supports a
    wide range of synthesis, effect and analysis algorithms. However, available operations are
    limited to those implemented explicitly as Unit Generators (UGens). Since UGens are written in
    C/C++ and loaded as plugins during the SuperCollider server boot process, it is impossible to
    live code UGens, which limits the user to creating sound as a composition of existing UGens
    during a performance.  Many of the vector operations required for efficiently creating complex
    audio effects are notably missing or tedious to use. To overcome this, we present ScPy, a UGen
    which embeds Python within SuperCollider to enable the use of the optimized vector operations
    provided by the NumPy library.

\end{abstract}

\section{Introduction}

Although a number of open-source projects which interface other languages with SuperCollider exist,
they are primarily SuperCollider clients, which have features to control synths and send other
automation messages. One notable exception is OctaveSC, which embeds GNU Octave within
SuperCollider. However, it is designed primarily for performing operations on control rate data
arrays and its author notes performance issues which prevent it from being used on audio rate data.

One of the goals of this project is to enable more flexible experimentation with FFT and phase
vocoder based operations. SuperCollider contains a total of 32 built in phase vocoder operations,
which provide the ability to produce many common effects. There are additionally some extra
operations available as user created extensions. However, many conceivable effects are impossible.

ScPy enables users to overcome these limitations. By embedding the Python programming language
within SuperCollider, it is possible to do numerical processing operations which would be too slow
to perform in real time with SuperCollider alone. Through access to the NumPy library, previously
difficult or impossible phase vocoder operations become trivial. Although our usage of this library
for testing purposes has mostly focused on FFT based operations, support exists for passing
arbitrary data buffers to Python, which can be used to process numerical data for many other
purposes as well.

\section{Methodology}

During the initial planning stage of the project, we wanted to essentially provide a set of
fundamental operations on spectral data which could be composed together to perform any conceivable
effect. It was also a requirement that these operations would be syntactically concise in order to
enable easy and fast experimentation. Ideally, the system would be able to handle complex effects
in real-time.

Although designing a domain-specific language (DSL) for this task was briefly considered, we
eventually decided to use Python for a number of reasons. First, Python includes an excellent C API
which makes embedding it within other languages trivial in comparison to many other options.
Second, Python's syntax is easily readable and concise -- it clearly adheres to our goals. Finally,
Python has a massive number of community built libraries for performant scientific and numerical
computing. With a few small exceptions, these libraries contained every one of the fundamental
spectral operations we could think of.

It may seem as though embedding an interpreted language within an interpreted language could offer
no performance advantage. There is certainly some performance hit when switching languages and
transferring data. However, the biggest advantage comes not strictly from performance but from the
diversity of operations which become available with highly optimized implementations. NumPy is used
internally by ScPy for handling data arrays, so the entire NumPy library of mathematical functions
comes at no cost.  Any other Python library can easily be imported as well. Limited experiments
have been done with the SciPy library. It would also be possible to use Theano to do processing on
the GPU for extra performance. Implementing an equivalent to these popular Python libraries in pure
SuperCollider is certainly possible, but it would require a very large amount of work and would see
little widespread adoption since SuperCollider is ultimately a niche language.

\section{Usage}

Use of ScPy is through two UGens: \texttt{Py} and \texttt{PyOnce}. Both have one required argument,
a block of Python code. A map of variable names may optionally be provided to bind SuperCollider
variables to Python variables.  Additionally, there is an optional \texttt{DoneAction} argument which
specifies an action to occur in SuperCollider after the Python code has finished executing.

\begin{listing}[H] \inputminted[linenos=true]{SuperCollider}{../examples/template.sc}
\caption{SuperCollider boilerplate for no-op FFT effect with ScPy.} \label{lst:boilerplate}
\end{listing}

When listing \ref{lst:boilerplate} is run, an anonymous synth is created which will take in audio
from an external source, perform a STFT, process the spectral data with ScPy, perform an inverse
STFT, then finally output the resulting audio. Lines 8-9 contain Python code defining the
processing function where operations are to be inserted. Further examples will replace only those
lines.

\section{Implementation}

Since UGens may only be used as part of a synth, \texttt{PyOnce} is provided as a wrapper which
hides this detail from the user and allows the execution of Python code outside of a synth. It is
useful for doing initialization work.  Since the current implementation executes Python code in a
single global namespace, a typical usage patten is for state to be defined in a \texttt{PyOnce}
block which is later accessed and modified in a \texttt{Py} block.

ScPy is written mostly in C++, but with a small SuperCollider class library to connect the C++
back-end, and a small Python library that provides some useful operations. Due to differences
between these languages, connecting them was somewhat awkward in some cases.

One issue involves the process of passing data between SuperCollider and C++. Since UGens are
designed to work primarily with data streams, adding support for passing in many types of data came
with a number of challenges. Essentially, all data must be serialized into an array of floats.
Strings must be converted to an array of ASCII character codes, and prepended with their length.
Arguments passed through to the Python code can have one of many types, and therefore type
information must be encoded as well. However, just passing the type of a variable is not enough to
know what we can do with it, since we also want to know if it inherits from a class we can use.

\section{Future Work}

At the time of writing, ScPy only supports handling audio rate data via SuperCollider's
\texttt{Buffer} objects. In order to make some use cases more ergonomic, ScPy could certainly be
extended to support raw audio rate input. For example, this would make time domain audio data even
simpler to work with than frequency domain data is currently. It would also enable SuperCollider's
FFT UGen to be easily re-implemented in Python for greater flexibility.

\section{Conclusion}

\section{Bibliography}

\end{document}
