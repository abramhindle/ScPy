\documentclass{article}

\usepackage{fancyhdr} \usepackage{hyperref}

\pagestyle{fancy}

\begin{document}

\title{ScPy -- A SuperCollider extension for performing numerical computation via embedded Python}
\author{Noah Weninger \and Abram Hindle}

\maketitle

\begin{abstract}
    SuperCollider, a language for sound synthesis and algorithmic composition of audio,
    supports a wide range of synthesis, effect and analysis algorithms. However, available
    operations are limited to those implemented explicitly as Unit Generators (UGens). Since UGens
    are written in C/C++ and loaded as plugins during the SuperCollider server boot process, it is
    impossible to live code UGens, which limits the user to creating sound as a composition of
    existing UGens during a performance. Many of the vector operations required for efficiently
    creating complex audio effects are notably missing or tedious to use. To overcome this, we
    present ScPy, a UGen which embeds Python within SuperCollider to enable the use of the
    optimized vector operations provided by the NumPy and SciPy libraries.
\end{abstract}

\section{Introduction}

\section{Literature Review}

\section{Implementation}

\section{Evaluation}

\section{Conclusion}

\section{Bibliography}

\end{document}
