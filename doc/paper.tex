% vim: tw=99

\documentclass{article}

\usepackage{fancyhdr} \usepackage{hyperref} \usepackage{listings}

\pagestyle{fancy}

\begin{document}

\title{ScPy: A SuperCollider Extension for Performing Numerical Computation via Embedded Python}

\author{Noah Weninger \and Abram Hindle}

\maketitle

\begin{abstract}

SuperCollider, a language for sound synthesis and algorithmic composition of audio, supports a wide
range of synthesis, effect and analysis algorithms. However, available operations are limited to
those implemented explicitly as Unit Generators (UGens). Since UGens are written in C/C++ and
loaded as plugins during the SuperCollider server boot process, it is impossible to live code
UGens, which limits the user to creating sound as a composition of existing UGens during a
performance.  Many of the vector operations required for efficiently creating complex audio effects
are notably missing or tedious to use. To overcome this, we present ScPy, a UGen which embeds
Python within SuperCollider to enable the use of the optimized vector operations provided by the
NumPy library.

\end{abstract}

\section{Introduction}

Although a number of open-source projects which interface other languages with SuperCollider exist,
they are primarily SuperCollider clients, which have features to control synths and send other
automation messages.  One notable exception is OctaveSC, which embeds GNU Octave within
SuperCollider.  However, it is designed primarily for performing operations on control rate data
arrays and its author notes performance issues which prevent it from being used on audio rate data.

One of the goals of this project is to enable more flexible experimentation with FFT and phase
vocoder based operations.  SuperCollider contains a total of 32 built in phase vocoder operations,
which provide the ability to produce many common effects. There are additionally some extra
operations available as user created extensions. However, many conceivable effects are impossible.

ScPy allows users to overcome these limitations. By embedding the Python programming language
within SuperCollider, it is possible to do numerical processing operations which would be too slow
to perform in real time with SuperCollider alone. Through access to the NumPy library, previously
difficult or impossible phase vocoder operations become trivial.  Although our usage of this
library for testing purposes has mostly focused on FFT based operations, support exists for passing
arbitrary data buffers to Python, which can be used to process numerical data for many other
purposes as well.

It may seem as though embedding an interpreted language within an interpreted language could offer
no performance advantage. There is certainly some performance hit when switching languages and
transferring data. However, the biggest advantage comes not strictly from performance but from the
diversity of operations which become available with highly optimized implementations. NumPy is used
internally by ScPy for handling data arrays, so the entire NumPy library of mathematical functions
comes at no cost.  Any other Python library can easily be imported as well. Limited experiments
have been done with the SciPy library. It would also be possible to use Theano to do processing on
the GPU for extra performance. Implementing an equivalent to these popular Python libraries in pure
SuperCollider is certainly possible, but it would require a very large amount of work and would see
little widespread adoption since SuperCollider is ultimately a niche language.

\section{Literature Review}

\section{Implementation}

Use of ScPy is through two UGens: \texttt{Py} and \texttt{PyOnce}. Both have one required argument,
a block of Python code. A map of variable names may optionally be provided to bind SuperCollider
variables to Python variables.  Additionally, there is a \texttt{DoneAction} argument which
specifies an action to occur in SuperCollider after the Python code has finished executing.

Since UGens may only be used as part of a synth, \texttt{PyOnce} is provided as a wrapper which
hides this detail from the user and allows the execution of Python code outside of a synth. It is
useful for doing initialization work.  Since the current implementation executes Python code in a
single global namespace, a typical usage patten is for state to be defined in a \texttt{PyOnce}
block which is later accessed and modified in a \texttt{Py} block.

\section{Evaluation}

Example usage:\\
This snippet will transform any input into an infinite falling Shepard tonerlike
sound.
\lstinputlisting[
    basicstyle=\footnotesize\ttfamily,
    firstline=3]{../test.sc}

\section{Future Work}

At the time of writing, ScPy only supports handling audio rate data via SuperCollider's
\texttt{Buffer} objects. In order to make some use cases more ergonomic, ScPy could certainly be
extended to support raw audio rate input. For example, this would make time domain audio data even
simpler to work with than frequency domain data is currently. It would also enable
SuperCollider's FFT UGen to be easily re-implemented in Python for greater flexibility.

\section{Conclusion}

\section{Bibliography}

\end{document}
